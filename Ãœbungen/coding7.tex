% load document class mhexsheet & select language as option (german or english)
\documentclass[german]{mhexsheet}

% set the parameters for the current exercise sheet with \exerciseSetup (key-value interface)
\exerciseSetup{
  lecture      = Einführung in die Informations- und Codierungstheorie,
  lectureshort = Codierungstheorie, % optional short name for the footer
  semester     = SS 2015,
  deadline     = {20.7.2015},
  lecturer     = Michael Helmling,
  operator     = Carolin Torchiani,
  homepage     = {http://userpages.uni-koblenz.de/~helmling/coding},
  sheetnumber  = 13,
  %solution, % solution (or solution=true) enables output of solution environments. solution=false (default) disables it,
  % logo = \includegraphics[width=5cm]{image.png}, % can override the standard logo
  % logowidth = 7cm % ... and also its width
}
\usepackage{cleveref}
\usepackage[
  binary-units=true,
  per-mode=symbol,
  exponent-product=⋅,
  locale=DE]{siunitx}
\usepackage{amssymb}
\usepackage[math,fonts=false]{mh_basic}
% == local commands 
\setcounter{MaxMatrixCols}{20}
\newcommand\A{\mathcal A}
\renewcommand\B{\operatorname{B}}
\renewcommand\d{\operatorname{d}}
\DeclareMathOperator{\w}{w}
\newcommand{\Mat}{\operatorname{Mat}}
\renewcommand{\t}[1]{\texttt{#1}}
\renewcommand{\(}{\left(}
\renewcommand{\)}{\right)}
\renewcommand\P{\mathcal P}
\newcommand\pErr{p_\mathrm{Err}}
\newcommand\D{D_{\mathrm{KL}}}

\newcommand{\length}{l}
\newcommand{\mc}{\mathcal}
\newcommand{\mb}{\mathbf}
\newcommand{\ol}[1]{[#1]}
\newcommand{\ov}[1]{\overline{#1}}
\newcommand{\id}{\operatorname{id}}
\newcommand{\Sy}{\mathbb S}
\newcommand{\GF}{\operatorname{GF}}
\newcommand{\Ce}{\mathbb C}
\newcommand{\Primes}{\mathbb P}
\newcommand{\Nk}{\operatorname {Nk}}
\newcommand{\Nr}{\operatorname {Nr}}
\newcommand{\AES}{\mathrm{AES}}
\newcommand{\ML}{\mathrm{ML}}
\newcommand{\expand}{\texttt{KeyExpansion}}
\newcommand{\size}{\operatorname{size}}
\newcommand{\parf}{\operatorname{par}}
\newcommand{\im}{\operatorname{im}}
\newcommand{\K}{\mathcal K}
\DeclareMathOperator\Pot{Pot}
\usepackage{tikz}
\renewcommand\C{\mathcal C}
\tikzset{
  extend/.style={shorten >=-#1,shorten <=-#1},
  extend/.default=5mm,
  shorten/.style={shorten >=#1,shorten <=#1},
  shorten/.default=5mm,
  wobbly/.style={decorate,decoration={snake,segment length=15pt,amplitude=0.5pt}},
  brace/.style={decorate,decoration={brace,#1}},
  box/.style={draw,fill=#1,rounded corners,minimum height=15pt},
  circ/.style={circle,inner sep=0mm,minimum size=2mm,draw},
  dot/.style={circle,inner sep=0mm,minimum size=2mm,fill}
}
% ==================

\setmathfont{texgyretermes-math.otf}
\setmathfont[range={"29F5}]{XITS Math}
\setmathfont[range={}]{texgyretermes-math.otf}
\usepackage{booktabs}

\begin{document}
\maketitle

\begin{exercise}
  Sei $\C$ der $(7,4)$-Hamming-Code. Bestimmen Sie für die bei Übertragung aus dem BSC erhaltenen Signale $y^1 = (1,1,0,0,1,1,0)$, $y^2=(1,1,1,0,1,1,0)$, und $y^3 = (1,1,1,1,1,1,0)$ jeweils das ML-Codewort.
\end{exercise}
\begin{exercise}
  \begin{enumerate}
     \item Beweisen Sie, dass es keinen $(7,3)$-Code mit Minimaldistanz $5$ gibt.
     \item \emph{(Für Ambitionierte!)} Beweisen Sie, dass es keinen $(90,78)$-Code mit Minimaldistanz $5$ gibt.
   \end{enumerate}
   \emph{Hinweis zu b):} Rechnen Sie nach, dass die Kugelpackungsgleichung erfüllt ist. Angenommen es gäbe einen solchen Code $\C$; oBdA (Translationsinvarianz des Hamming-Abstands) liegt der Nullvektor in $\C$. Betrachten Sie $V = \{v∈\GF(2)^{90}\colon v_1 = v_2 = 1, \w(v) = 3\}$ und $M = \{x∈\C\colon x_1 = x_2 = 1, \w(c) = 5\}$. Berechnen Sie
   \[ \abs{\{ (v, x) ∈ V×M\colon ⟨v,x⟩ = 1\}}\]
   sowohl über $\sum_{v∈V} \abs{\{x∈M\colon ⟨v,x⟩=1\}}$ als auch über $\sum_{x∈M} \abs{\{v∈V\colon ⟨v,x⟩=1\}}$.
\end{exercise}

\begin{exercise}
  Sei $\C$ ein linearer $(N,K)$-Blockcode. Durch \emph{Erweiterung} von $\C$ erhalten wir einen Code $\hat\C$ der Länge $N+1$, indem wir an jedes Codewort $x=x_1\dotsm x_N∈\C$ ein zusätzliches Bit $x_{N+1}$ anhängen, so dass $\sum_{i=1}^{N+1} x_i = 0$ erfüllt ist. Zeigen Sie:
  \begin{enumerate}
    \item $\hat\C$ ist ein linearer $(N+1,K)$-Code. Bestimmen Sie dazu eine Kontrollmatrix $\hat H$ von $\hat\C$, ausgehend von der Kontrollmatrix $H$ von $\C$.
    \item Ist $\d(\C)$ ungerade, so gilt $\d(\hat\C) = \d(\C) + 1$.
  \end{enumerate}
\end{exercise}

\begin{exercise}
  Sei $\C$ ein linearer $(N,K)$-Code. Zeigen Sie:
  \begin{enumerate}
    \item $\(\C^⊥\)^⊥ = \C$.
    \item Ist $G$ eine Generatormatrix der Form $G=(I_K\;M)$, wobei $I_n$ für $n∈ℕ$ die $n×n$-Einheitsmatrix ist, dann ist $H = (M^T\; I_{N-K})$ eine Kontrollmatrix für $\C$.
    \item Wie lässt sich die Bedingung aus b) bei beliebigem $G$ herstellen?
  \end{enumerate}
\end{exercise}
\end{document}