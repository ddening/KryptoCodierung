% load document class mhexsheet & select language as option (german or english)
\documentclass[german]{mhexsheet}

% set the parameters for the current exercise sheet with \exerciseSetup (key-value interface)
\exerciseSetup{
  lecture      = Einführung in die Kryptographie,
  lectureshort = Kryptographie, % optional short name for the footer
  semester     = SS 2015,
  deadline     = {27.4.2015, 12:00 Uhr},
  date  = {21.4.2015},
  lecturer     = Carolin Torchiani,
  operator     = Michael Helmling,
  %homepage     = www.google.de,
  sheetnumber  = 2,
  %solution, % solution (or solution=true) enables output of solution environments. solution=false (default) disables it,
  % logo = \includegraphics[width=5cm]{image.png}, % can override the standard logo
  % logowidth = 7cm % ... and also its width
}

\usepackage[math,fonts=false]{mh_basic}


\begin{document}
\maketitle

\inclass
\begin{exercise}[title = Gruppen]
 \begin{enumerate}
  \item Sei $n \in \N$. Beweisen Sie, dass $n\Z = \{nk \mid k \in \Z\}$ eine Untergruppe von $\Z$ ist.
   \item Seien $(G_1, \circ)$ und $(G_2, *)$ Gruppen und sei $f: G_1 \rightarrow G_2$ ein Gruppenhomomorphismus. Beweisen Sie, dass das Urbild des neutralen Elements von $G_2$ eine Untergruppe von $G_1$ ist. 
  \item $\mathbb S_4$ ist die Gruppe der bijektiven Abbildungen $\{1, 2, 3, 4\} \rightarrow \{1, 2, 3, 4\}$. Zeigen Sie, dass die Untergruppe
  \[U = \left\langle \begin{pmatrix} 1 & 2 & 3 & 4 \\ 2 & 3 & 4 & 1 \end{pmatrix} \right\rangle \subset \mathbb S_4\] 
  zyklisch ist mit Ordnung $4$. Geben Sie einen Isomorphismus zu $\Z_4$ an.
%   \item Sei $H$ eine zyklische Gruppe von Primzahlordnung und $h \in H$ beliebig. Bestimmen Sie die von $h$ erzeugte Untergruppe.
 \end{enumerate}
\end{exercise}


\begin{exercise}[title = Teilbarkeit]
 \begin{enumerate}
  \item Wie viele Elemente hat $\Z_{36}^*$?
  \item Bestimmen Sie alle Elemente von $\Z_{36}^*$ (anders ausgedrückt: alle Einheiten von $\Z_{36}$) und alle Nullteiler von $\Z_{36}$.
  \item Sei $R$ ein endlicher kommutativer Ring mit Eins. Beweisen Sie: Jedes Element von $R$ ist Einheit oder Nullteiler. (Gilt die Aussage auch, wenn $R$ undendlich viele Elemente besitzt?) 
 \end{enumerate}
\end{exercise}


\begin{exercise}[title = Euklidischer Algorithmus]
 \begin{enumerate}
  \item Berechnen Sie einen größten gemeinsamen Teiler von $588$ und $315$. Stellen Sie ihn als Linearkombination von $588$ und $315$ dar.
  
  \item Berechnen Sie das Inverse von $[19]$ in $\Z_{36}^*$.
  \item Lösen Sie, falls möglich, die folgenden Kongruenzgleichungssysteme in $\Z$:
 \begin{enumerate}[columns = 2]
  \item $3x \equiv \; 7 \mod 16$
  \item $4x \equiv \; 7 \mod 16$
 \end{enumerate}
 \end{enumerate}
\end{exercise}
% 
% \begin{exercise}[title = Kongruenzgleichungssysteme]
% Lösen Sie, falls möglich, die folgenden Kongruenzgleichungssysteme in $\Z$:
%  \begin{enumerate}[columns = 3]
%   \item \begin{align*} 
%          3x \equiv & \; 7 \mod 16
%         \end{align*}
%   \item \begin{align*} 
%          4x \equiv & \; 7 \mod 16
%         \end{align*}
%  \item \begin{align*}
% 	  x \equiv & \; 5 \mod 81 \\
% 	  x \equiv & \; 7 \mod 11
%        \end{align*}
%  \end{enumerate}
% \end{exercise}

% 
% \begin{exercise}[title = Polynomring]
%   Sei $R$ ein kommutativer Ring mit Eins. Ein \emphex[Polynom]{Polynom in einer Variablen über $R$} ist ein Ausdruck der Form
%  \[a_nt^n + a_{n-1}t^{n-1} + \dotsc + a_1t +a_0\]
%  mit \emph{Koeffizienten} $a_n$, \dots, $a_0 \in R$ und $n \in \N$. Wir fassen dabei $t$ als \emphex{formale Variable} auf. (Natürlich könnte man hier auch jeden anderen Buchstaben als formale Variable wählen.) Wir definieren $R[t]$ als die Menge aller Polynome über $R$. 
%  
%  Sind $f = \sum_{k=0}^{n_f} a_kt^k$ und $g = \sum_{k_0}^{n_g} b_kt^k$ Polynome, so definieren wir ihre Summe $f+g$ und ihr Produkt $f \cdot g$ als die Polynome
%  \begin{equation}
%  \begin{aligned}
%   && f + g & := \sum_{k = 0}^{n_f} (a_k + b_k) t^k \\
%   &\text{und} & f \cdot g& := \sum_{m = 0}^{n_f + n_g} \left(\sum_{k + l = m} a_k + b_l \right) t^m, 
%  \end{aligned}
%  \end{equation}
%  wobei $t^0 \in R[t]$ mit $1 \in R[t]$ identifiziert wird.
%  
%  Beachten Sie, dass die Multiplikation von Polynomen genauso eingeführt wurde, wie man es erwarten würde, wenn man diese formalen Summen naiv ausmultipliziert; dann wäre nämlich
%   \begin{align}
%    \left(\sum_{k=0}^{n_f} a_kt^k\right) \cdot \left(\sum_{l = 0}^{n_g} b_lt^l\right) & = (a_0 + a_1t + a_2t^2 + \dotsc) \cdot (b_0 + b_1t + b_2t^2 + \dotsc) \\
%    & = a_0b_0 + (a_0b_1 + a_1b_0)t + (a_0b_2 + a_1b_1 + a_2b_0)t^2 + \dotsc \\
%    & = \sum_{m = 0}^{n_k + n_l} \left(\sum_{k + l = m} a_k + b_l \right) t^m,
%   \end{align}
%  was genau die obige Definition ist. 
%  
%  Ist $K$ ein Körper, so ist $K[t]$ ein sogenannter euklidischer Ring. Damit ist der Begriff eines \enquote{größten gemeinsamen Teilers} zweier Polynome definiert und der (erweiterte) euklidische Algorithmus kann zur Berechnung verwendet werden. 
%  
%  \begin{enumerate}
%   \item Wir betrachten den Polynomring $\Q[t]$. 
%   \begin{enumerate}
%   \item Berechnen Sie $(3t^3 + 5t + 1) \cdot (5t^2 + 4)$.
%   \item Berechnen Sie den größten gemeinsamen Teiler von $f = t^4 - 1$ und $g = 3t^3 + 4t^2 + 3t + 4)$. Stellen Sie ihn in der Form $af + bg$ mit $a, b \in \Q[t]$ dar.
%   \end{enumerate}
%   \item Nun betrachten wir den Polynomring $\Z_2[t]$. 
%   \begin{enumerate}
%   \item Berechnen Sie $(t^2 + t + 1) \cdot (t^4 + t + 1)$.  
%   \item Bestimmen Sie einen größten gemeinsamen Teiler von $f = t^5 + t + 1$ und $g = t^4 + t^2 + 1$. Stellen Sie ihn in der Form $af + bg$ mit $a, b \in \Z_2[t]$ dar.
%  \end{enumerate}
%  \end{enumerate}
% \end{exercise}
\takehome

\begin{exercise}[title = Schwächen der Enigma]
 Bestimmen Sie den prozentualen Anteil der Permutationen auf einem Alphabet der Länge $26$, die involutorisch und fixpunktfrei sind. 
 
 Zur Erinnerung: Eine Permutation $\sigma \in \mathbb S(M)$ der Menge $M$ ist eine bijektive Abbildung $\sigma: M \rightarrow M$. Sie heißt involutorisch, falls $\sigma^2(m) = m$ für alle $m \in M$ gilt, und fixpunktfrei, wenn $\sigma(m) \neq m$ für alle $m \in M$ erfüllt ist.
\end{exercise}


\begin{exercise}[title = Vigenère-Chiffre]
 Modellieren Sie die Vigenère-Chiffre als Kryptosystem.
\end{exercise}

\begin{exercise}[title = Verschlüsselungsfunktionen]
 Zeigen Sie, dass die Verschlüsselungsfunktionen $E_e\colon \mathcal P \rightarrow \mathcal C$ eines Kryptosystems injektiv sind.
\end{exercise}

\begin{exercise}[title = Satz von Bayes]
\begin{enumerate}
 \item Seien $(\Omega, P)$ ein Wahrscheinlichkeitsraum und $X\colon \Omega \rightarrow \mathcal A_X$ und $Y\colon \Omega \rightarrow \mathcal A_Y$ zwei Zufallsvariablen. Zeigen Sie:
 \begin{enumerate}
  \item Satz von Bayes:
  \[P_{X|Y \in S}(T) = P_{Y|X \in T}(S) \frac{P_X(T)}{P_Y(S)}\]
  für alle Ereignisse $T \subseteq \mathcal A_X$, $S \subseteq \mathcal A_Y$ mit $P_X(T), P_Y(S) > 0$. 
  \item $X$ und $Y$ sind genau dann unabhängig, wenn für alle $T \subseteq \mathcal A_X$ und $S \subseteq \mathcal A_Y$ gilt:
  \[P_Y(S) = 0 \text{ oder } P_{X|Y \in S}(T) = P_X(T).\]
 \end{enumerate}
\item Bei Joe wird ein Test auf eine gefährliche Krankheit durchgeführt. $1\%$ der Bevölkerung in Joes Alter mit vergleichbarer Lebensführung leidet an der Krankheit. In $95\%$ der Fälle ergibt der Test für erkrankte Personen ein positives Ergebnis, in $95\%$ der Fälle ergibt der Test für gesunde Personen ein negatives Ergebnis. Joe erhält ein positives Testergebnis. Mit welcher Wahrscheinlichkeit ist er tatsächlich krank? 

Einige Tipps zur Vorgehensweise: Die Ergebnismenge ist $\Omega = \{(a, b)\,|\,a, b \in \{0, 1\}\}$, ein Ergebnis $(a,b) \in \Omega$ wird dabei wie folgt interpretiert:
\begin{center}
\begin{tabular}{lllll}
 $a = 0$ & Joe ist gesund. & \quad &$b = 0$ & Der Test ist negativ. \\
 $a = 1$ & Joe ist krank. & \quad &$b = 1$ & Der Test ist positiv.  
\end{tabular}
\end{center}
Auf $\Omega$ sind zwei Zufallsvariablen definiert, nämlich $A\colon \Omega \rightarrow \{0, 1\}$, $(a, b) \mapsto a$, und $B\colon \Omega \rightarrow \{0, 1\}$, $(a, b) \mapsto b$. Die Zufallsvariable $A$ gibt über den Krankheitszustand Auskunft, $B$ über das Testergebnis. Die Verteilungen $P: \Omega \rightarrow [0, 1]$, $P_A: \{0, 1\} \rightarrow [0, 1]$ und $P_B: \{0, 1\} \rightarrow [0, 1]$ seien durch die oben im Text angegebenen Prozentzahlen gegeben.

\begin{enumerate}
\item Bestimmen Sie Wahrscheinlichkeiten $P_A(a)$ für $a =0, 1$. 
\item Bestimmen Sie die bedingten Wahrscheinlichkeiten $P_{B|A =a}(b)$ für $a, b \in \{0, 1\}$.
\item Bestimmen Sie $P(a, b)$ in Abhängigkeit von $P_A(a)$ und $P_{B|A = a}(b)$ für $a, b \in \{0, 1\}$.
\item Bestimmen Sie $P_{A| B = 1}(1)$ mit dem Satz von Bayes.
\end{enumerate}
 \end{enumerate}
\end{exercise}
% 
% \begin{exercise}[title = Zufallsvariablen]
% \begin{enumerate}
%  \item Was trifft zu: Wenn $X$ und $Y$ unabhÀngige Zufallsvariablen sind, dann gilt:
% \begin{itemize}
% \item[a)]  $ P((X \in A) \cap (Y \in B)) = P(X \in A ) + P(Y \in B) $ \antwbox % falsch
% \item[b)]  $ P((X \in A) \cup (Y \in B)) = P(X \in A ) + P(Y \in B) $ \antwbox % falsch
% \item[c)]  $ P((X \in A) \cup (Y \in B)) = P(X \in A ) + P(Y \in B)  
%             - P((X \in A) \cap (Y \in B)) $ \antwbox % richtig
% \item[d)]  $ P((X \in A) \cap (Y \in B)) = P(X \in A ) \cdot  P(Y \in B) $ \antwbox % richtig
% \item[e)]  $ P((X \in A) \cup (Y \in B)) = P(X \in A ) \cdot P(Y \in B) $ \antwbox  % falsch
% \end{itemize}
% \item bla
% \end{exercise}


\end{document}