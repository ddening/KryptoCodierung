% load document class mhexsheet & select language as option (german or english)
\documentclass[german]{mhexsheet}

% set the parameters for the current exercise sheet with \exerciseSetup (key-value interface)
\exerciseSetup{
  lecture      = Einführung in die Informations- und Codierungstheorie,
  lectureshort = Codierungstheorie, % optional short name for the footer
  semester     = SS 2015,
  deadline     = {13.7.2015},
  lecturer     = Michael Helmling,
  operator     = Carolin Torchiani,
  homepage     = {http://userpages.uni-koblenz.de/~helmling/coding},
  sheetnumber  = 12,
  %solution, % solution (or solution=true) enables output of solution environments. solution=false (default) disables it,
  % logo = \includegraphics[width=5cm]{image.png}, % can override the standard logo
  % logowidth = 7cm % ... and also its width
}
\usepackage{cleveref}
\usepackage[
  binary-units=true,
  per-mode=symbol,
  exponent-product=⋅,
  locale=DE]{siunitx}
\usepackage{amssymb}
\usepackage[math,fonts=false]{mh_basic}
% == local commands 
\setcounter{MaxMatrixCols}{20}
\newcommand\A{\mathcal A}
\renewcommand\B{\operatorname{B}}
\renewcommand\d{\operatorname{d}}
\DeclareMathOperator{\w}{w}
\newcommand{\Mat}{\operatorname{Mat}}
\renewcommand{\t}[1]{\texttt{#1}}
\renewcommand{\(}{\left(}
\renewcommand{\)}{\right)}
\renewcommand\P{\mathcal P}
\newcommand\pErr{p_\mathrm{Err}}
\newcommand\D{D_{\mathrm{KL}}}

\newcommand{\length}{l}
\newcommand{\mc}{\mathcal}
\newcommand{\mb}{\mathbf}
\newcommand{\ol}[1]{[#1]}
\newcommand{\ov}[1]{\overline{#1}}
\newcommand{\id}{\operatorname{id}}
\newcommand{\Sy}{\mathbb S}
\newcommand{\GF}{\operatorname{GF}}
\newcommand{\Ce}{\mathbb C}
\newcommand{\Primes}{\mathbb P}
\newcommand{\Nk}{\operatorname {Nk}}
\newcommand{\Nr}{\operatorname {Nr}}
\newcommand{\AES}{\mathrm{AES}}
\newcommand{\ML}{\mathrm{ML}}
\newcommand{\expand}{\texttt{KeyExpansion}}
\newcommand{\size}{\operatorname{size}}
\newcommand{\parf}{\operatorname{par}}
\newcommand{\im}{\operatorname{im}}
\newcommand{\K}{\mathcal K}
\DeclareMathOperator\Pot{Pot}
\usepackage{tikz}
\renewcommand\C{\mathcal C}
\tikzset{
  extend/.style={shorten >=-#1,shorten <=-#1},
  extend/.default=5mm,
  shorten/.style={shorten >=#1,shorten <=#1},
  shorten/.default=5mm,
  wobbly/.style={decorate,decoration={snake,segment length=15pt,amplitude=0.5pt}},
  brace/.style={decorate,decoration={brace,#1}},
  box/.style={draw,fill=#1,rounded corners,minimum height=15pt},
  circ/.style={circle,inner sep=0mm,minimum size=2mm,draw},
  dot/.style={circle,inner sep=0mm,minimum size=2mm,fill}
}
% ==================

\setmathfont{texgyretermes-math.otf}
\setmathfont[range={"29F5}]{XITS Math}
\setmathfont[range={}]{texgyretermes-math.otf}
\usepackage{booktabs}

\begin{document}
\maketitle
\begin{exercise}
  Sei $\C$ ein $(N,K)$-Blockcode, $X$ eine Zufallsvariable mit Alphabet $\C$ (also ein zufälliges Codewort), $\K=(B, P_{\K,0}, P_{\K,1})$ ein Kanal und $(X,Y)$ das entsprechende Paar aus Kanalein- und Ausgabe. Ein \emph{Maximum A-Posteriori-(MAP-)Decodierer} $D_\mathrm{MAP}$ entscheidet sich für ein unter Berücksichtigung der empfangenen Kanalausgabe $\{Y=y\}$ mit $y∈B^N$ am wahrscheinlichsten gesendetes Codewort:
  \[ D_\mathrm{MAP}(y) = \argmax_{x∈\C} P(X=x∣Y=y) \]
  Zeigen Sie: Bei gleichverteiltem $X$ ist jeder ML-Decodierer auch ein MAP-Decodierer.
\end{exercise}

\begin{exercise}
  Zeigen Sie: Der Hamming-Abstand $\d(x,y) \coloneqq \abs{\{i\colon x_i ≠ y_i\}}$ ist eine translationsinvariante Metrik auf $\GF(2)^N$.
\end{exercise}

\begin{exercise}
  Die \emph{Global Trade Item Number (GTIN)}, auch \emph{European Article Number (EAN)} ist eine 13-stellige Zahl zur international eindeutigen Kennzeichnung von Handelsprodukten, die in der Regel in Form eines Strichcodes auf Produkte gedruckt wird. Die GTIN enthält einen Code zur Fehlererkennung:  nur die ersten 12 Ziffern enthalten Informationen zu Land, Unternehmen und Artikelnummer, während an 13.\ Stelle eine \emph{Prüfziffer} steht, die so gewählt wird, dass
  \[ x_1 + 3 x_2 + x_3 + 3 x_4 + \dotsm + 3 x_{12} + x_{13} ≡ 0 \pmod{10}\]
  gilt, wenn $x_1\dotsm x_{13}$ die 13 Stellen der GTIN sind.
  \begin{itemize}
    \item Wie viele Fehler (falsch gelesene Ziffern) kann dieser Code \emph{korrigieren}?
    \item Wie viele Fehler kann er \emph{erkennen}?
    \item Wann kann eine \emph{Vertauschung} von zwei benachbarten Ziffern erkannt werden?
  \end{itemize}  
  Der GTIN-Code ist mit dem ISBN13-Code für Bücher identisch, der die alte 10-Stellige ISBN-Nummer abgelöst hat. Diese besaß ebenfalls eine Prüfziffer $x_{10} ∈ \{0,1,\dotsc,9,X\}$ (wobei $X$ für 10 steht), die nach der Bedingung
  \[ 10 x_1 + 9 x_2 + \dotsm + 2 x_9 + x_{10} \equiv 0 \pmod{11} \]
  gebildet wird. Zeigen Sie, dass ISBN10 \emph{jede} Vertauschung von zwei Ziffern erkennt!
\end{exercise}
 
\end{document}